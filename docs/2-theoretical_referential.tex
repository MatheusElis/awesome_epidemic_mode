%%%%%%%%%%%%%%%%%%%%%%%%%%%%%%%%%%%%%%%%%%%%%%%%%%%%%%%%%%%%%%%%%%%%%%%%%%%%%%%

\chapter{REFERENCIAL TEÓRICO}
\label{chap:ref_teo}

Este capítulo apresenta o referencial teórico necessário para a execução desse trabalho. Ele é dividido em quatro seções contendo os principais temas que fundamentam esse documento. A primeira seção introduz a teoria base de epidemiologia e a importância de modelos matemáticos em epidemiologia. A segunda parte introduz a teoria base de grafos e redes complexas necessária para a formulação e o entendimento do modelo discreto que iremos propor neste trabalho. A terceira seção introduz o modelo contínuo que será executado e toda a sua base teórica. A quarta e ultima seção introduz o modelo discreto que será executado e toda a sua base teórica.

\section{Modelos Matemáticos em Epidemiologia}
Os modelos são uma ferramenta extremamente poderosa para fazer representações da realidade e segundo \citeonline{Massad2004} podem ser definidos operacionalmente como uma representação conveniente de alguma coisa importante. Quando um modelo consiste de uma representação de componentes quantitativos, este modelo é chamado de \textit{modelo matemático}.

Um modelo matemático por definição é composto pelos seguintes itens:
\begin{itemize}
    \item \textbf{Variáveis}: quantidade de interesse que variam com o tempo, tais como o número de indivíduos susceptíveis a uma dada infecção;
    
    \item \textbf{Parâmetros}: quantidades que definem o comportamento dinâmico do sistema, tais como a probabilidade de infecção de uma doença;
    
    \item \textbf{Condições iniciais e de contorno}: valores iniciais das variáveis com o tempo ($t=0$, condições iniciais) ou com a idade ($a = 0$, condições de contorno).
\end{itemize}

\section{Grafos e Redes Complexas}


\section{Modelo Contínuo}


\section{Modelo Discreto}